\documentclass[a4paper]{article}
\usepackage[utf8]{inputenc}
\usepackage[italian]{babel}
\usepackage[T1]{fontenc}
\usepackage{amsmath,amssymb,amsthm}
\usepackage{enumerate}

% \hfuzz=100.0pt  % ignore paragraph lengths warnings
% \usepackage[hidelinks]{hyperref}

\title{Overengineered \\
\large Relazione del progetto per l'insegnamento di Programmazione}
\author{
  M. Girolimetto,
  A. Scrob,
  L. Tagliavini,
  S. Volpe
}

\date{
	Universit\`a di Bologna \\
  \today
}

\begin{document}

\maketitle

\section{Funzionalit\`a}

Overengineered \`e un videogioco a piattaforme a scorrimento laterale con mappa
illimitata implementato in \verb!C++!. L'esecuzione avviene all'interno di un terminale:
l'interfaccia \`e perci\`o realizzata tramite una grafica ASCII. La mappa \`e
genearata in modo incrementale tramite l'unione di frammenti presviluppati. La
possibilit\`a di scegliere tra pi\`u personaggi, unita alla variet\`a di nemici
ed oggetti, rende il sistema di combattimento una sintesi fra scontri corpo a
corpo e a distanza.

\section{Spartizione dei ruoli}

Il lavoro \`e stato diviso tra i membri del gruppo sia sotto il piano della
programmazione che delle figure di contorno richieste per la creazione di un videogioco.

\subsection{Programmazione}

Il programma si articola in pi\`u moduli, che sono rappresentati nel codice con
lo strumento linguistico dello spazio dei nomi (\emph{namespace}). \`E stato
designato un responsabile diverso per ciascuno di questi moduli.
\begin{itemize}
  \item \verb!Data!:
  \item \verb!Engine!:
  \item \verb!World!:
  \item \verb!Game!:
\end{itemize}

Date le limitazioni sull'uso della \emph{Standard Template Library} offerta dal
linguaggio, il progetto include un ulteriore modulo, denominato \verb!Nostd!, che emula
alcuni strumenti della libreria standard. Quest'ultimo \`e stato realizzato
congiuntamente dall'intera squadra.

\end{document}
