\documentclass[a4paper]{article}
\usepackage[utf8]{inputenc}
\usepackage[italian]{babel}
\usepackage[T1]{fontenc}
\usepackage{amsmath,amssymb,amsthm}
\usepackage{enumerate}
\usepackage{epigraph}

\title{Overengineered \\
\large Relazione del progetto per l'insegnamento di Programmazione}
\author{
  M. Girolimetto,
  A. Scrob,
  L. Tagliavini,
  S. Volpe
}

\date{
	Universit\`a di Bologna \\
  \today
}

\begin{document}

\maketitle

\epigraph{Il tempo che vi divertite a sprecare non è tempo sprecato.}
{\textit{Bertrand Russell}}

\section{Funzionalit\`a}

\emph{Overengineered} \`e un videogioco a piattaforme a scorrimento laterale con
mappa illimitata implementato in \verb!C++!. L'esecuzione avviene all'interno di
un terminale: l'interfaccia \`e perci\`o realizzata tramite una grafica ASCII.
La mappa \`e generata in modo incrementale tramite l'unione di frammenti
presviluppati. La possibilit\`a di scegliere tra pi\`u personaggi, unita alla
variet\`a di nemici ed oggetti, rende il sistema di combattimento una sintesi
fra scontri corpo a corpo e a distanza. Ciascun evento degno di nota viene
inoltre descritto al giocatore in forma didascalica in tempo reale.

\section{Spartizione dei ruoli}

Il lavoro \`e stato diviso tra i membri del gruppo sia sotto il piano della
programmazione che delle figure di contorno richieste per la creazione di un
videogioco.

\subsection{Programmazione}

Il programma, scritto usando \verb!C++11! come norma tenica, si articola in
pi\`u moduli, che sono rappresentati nel codice con lo strumento linguistico
dello spazio dei nomi (\emph{namespace}). \`E stato designato un responsabile
diverso per ciascuno di questi moduli.
\begin{itemize}
  \item \verb!Data! (S. Volpe): descrive la base di dati del gioco e gli enti in
    essa contenuti, come \verb!Scenery!, \verb!Hero!, \verb!Skill!. La lettura
    di tali informazioni avviene da disco.
  \item \verb!Engine! (L. Tagliavini): fornisce il nucleo \emph{software} per
    l'uso di grafica in tempo reale e audio. Offre strutture per il
    \emph{rendering}, la creazione di interfacce utente e la riproduzione della
    colonna sonora.
  \item \verb!World! (M. Girolimetto): genera una mappa di gioco dinamicamente
    espandibile, popolandola con entit\`a di vario genere. Il livello di
    difficolt\`a aumenta in relazione alle dimensioni della mappa.
  \item \verb!Game! (A. Scrob): combina le funzionalit\`a degli altri moduli in
    un unico \emph{game loop}. Implementa la logica di gioco, incluso il sistema
    di combattimento, e la navigazione tra menu diversi.
\end{itemize}

Le specifiche del progetto prevedono serie limitazioni sull'uso della
\emph{Standard Template Library}. I contenitori come \verb!vector!, \verb!list!
e \verb!map! e \verb!string!, per esempio, sono vietati. Per far fronte a tali
richieste, \`e stato necessario includere un ulteriore modulo, denominato
\verb!Nostd!. Esso emula alcuni strumenti della libreria standard, integrandoli
secondo le necessit\`a del progetto. \verb!Nostd! \`e stato realizzato
congiuntamente dall'intera squadra: a ciascun membro del gruppo \`e stata
affidata una parte delle sue intestazioni. In cima a ciascun \emph{file} \`e
comunque presente il nominativo del programmatore per esso responsabile.

\subsection{Miscellanei}

La cartella \verb!assets! raccoglie le risorse (file di estensioni \verb!.txt! e
\verb!.csv!) caricate dalla base di dati.
Esse sono classificabili in quattro categorie.

\begin{itemize}
  \item Progettazione (S. Volpe): creazione dei frammenti della mappa e delle
    statistiche di gioco.
  \item Grafica (L. Tagliavini): scelta di caratteri e colori per la
    rappresentazione di paesaggi, entit\`a e ritratti.
  \item Audio (M. Girolimetto): composizione della colonna sonora riprodotta
    nelle varie fasi del gioco.
  \item Testi (A. Scrob): nomi delle entit\`a e descrizioni dei personaggi
    giocabili.
\end{itemize}

\section{Scelte implementative}

I paradigmi di programmazione usati come modello nella stesura del codice
sorgente sono quello procedurale, orientato agli oggetti e generico. Il
linguaggio scelto ci ha permesso di coniugare l'uso di potenti strumenti
linguistici e concettuali per l'astrazione con accorgimenti di basso livello
per ottenere prestazioni relativamente soddisfacenti.

\subsection{Nostd}

\verb!Nostd! (re)implementa liste, vettori, stringhe, matrici e mappe, ovvero i
contenitori utili al resto del progetto. Fedelt\`a alla \emph{STL},
semplicit\`a e pragmatismo ne hanno guidato la progettazione. Proprio come la
libreria a cui si ispira, \verb!Nostd! supporta i tipi di iteratori appropriati:
questo porta a diversi vantaggi, fra cui la possibilit\`a di usare sia
l'istruzione \emph{range-for} che \verb!<algorithm>!. \verb!<allocator.hpp>!,
\verb!<concepts.hpp>! e \verb!<pair.hpp>!, infine, hanno funto da intestazioni
ausiliarie.

\subsection{Data}

Così come prescritto da B. Stroustrup (\guillemotleft A meno che esista una
buona ragione per non farlo, usate \verb!vector!.\guillemotright\footnote{
\label{note1} Vedi B. Stroustrup, \emph{C++.
Linguaggio, libreria standard, principi di programmazione}, 4. ed., trad. it. di
Giulia Maselli e Paolo Postinghel, Pearson Italia, Milano, 2015, p. 810.}), la
quasi totalità delle informazioni della base di dati viene memorizzata in
vettori del tipo appropriato. Fa eccezione \verb!Result!, le cui istanze sono
memorizzate in una lista per privilegiare inserimenti interni durante
l'iterazione su esse. Molte delle classi coinvolte, inoltre, appartengono ad
un'unica gerarchia circoscritta dal sottospazio dei nomi \verb!Data::Pawns!.
L'interazione con il disco \`e regolata da sovraccaricamenti degli operatori di
slittamento a sinistra (\verb!<<!) e a destra (\verb!>>!) specifici per ciascun
tipo di ente trattato dalla base di dati.

\subsection{Engine}

Tutte le schermate ad eccezione della scena di gioco sono composte da una serie
di elementi visuali contenuti in \verb!Engine::UI!. Essi vengono inseriti in una
struttura arborescente che ricorda quella di un \emph{Document Object Model}:
questo permette, oltre al riutilizzo del codice, una più netta separazione tra
contenuto e logica dei menu.
La schermata di gioco scorre la lista di frammenti del mondo vicini al giocatore
per mostrarli a schermo sovrapponendovi entit\`a e proiettili.

\subsection{World}

Sia la mappa di gioco che l'insieme degli agenti che la popolano sono
rappresentati da liste. In questo modo, vengono semplificate le operazioni di
accodamento ed eliminazione da effettuare con il progredire del gioco. Le
coordinate di ciascun agente sono espresse da un iteratore facente riferimento
alla matrice in cui esso si trova, nonch\'e dagli indici della riga e della
colonna relative.

\subsection{Game}

In fase di inizializzazione, \verb!Game! richiede la costruzione di una nuova
base di dati specificando i percorsi da cui recuperare le risorse necessarie.
In seguito, affinch\'e il modulo possa gestire il \emph{game loop}, ad ogni
aggiornamento dello stato della partita la lista degli agenti viene scorsa per
intero una ed una sola volta. Tale processo prevede spostamenti sulla mappa,
controlli di collisioni, modifiche alle statistice e distruzioni. Ogni
avvenimento rilevante per il flusso del gioco innesca la composizione dinamica
di un messaggio di commento da mostrare a schermo.

\subsection{Unit testing}

La natura interconnessa dei moduli del progetto ha fin da subito messo in luce
il bisogno di poter verificare il funzionamento del codice per garantirne
l'affidabilit\`a sul lungo periodo. A tal fine si fa uso della funzione
ausiliaria \verb!Nostd::test! e del comando \verb!make test! per compilare ed
eseguire tutti i file per il \emph{testing} d'unit\`a.

\end{document}
